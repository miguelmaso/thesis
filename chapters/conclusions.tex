
\chapter{Conclusions}
\label{conclusions}



%%%%%%%%%%%%%%%%%%%%%%%%%%%%%%%%%%%%%%%%%%%%%%%%%%%%%%%%%%%%

\section{Achievements}




The main objective of this thesis is the investigation of Finite Element formulations applied to large-scale water-related hazards. In the first part of the thesis, a revision of reduced models for the large scale has been done.

In chapter \ref{eulerian_sw} The FIC-FEM procedure has been extended to the shallow water equations. Unlike the FIC-based stabilizations for incompressible flows, the present procedure is applied to the coupled mass and momentum balance at the same time using the linearization matrix $\mathbf{A}_i$. It can be seen that this procedure casts to the classical FIC-stabilization for convection diffusion problems, taking the velocity as linearization term. The same procedure can be applied to develop stabilized formulations for compressible flows.

The present extension of the FIC-procedure to the shallow water equations uses the linearization matrix $\mathbf{A}_i$ for the flux terms to project the characteristic length. However, an alternative framework can be explored with the ASGS \cite{hughes1995,codina2008} formulation, which includes the linearization matrices of the viscous terms and reaction terms. Since the shallow water equations are dominated by the convective matrix $\mathbf{A}_i$, and thus are strictly hyperbolic, the present stabilization is enough to provide stability, as shown in Section \ref{sec:examples}.

The present stabilization provides two algorithmic constants, one for the global stabilization and other one for the shock capturing term. From our numerical experiments, we have chosen $\beta=0.01$ for the stabilization and $\alpha=1.0$ for the shock capturing.

Regarding the accuracy of the shock capturing and the dry domain model, one must notice that this method is not monotonic. Therefore, like in many other stabilized methods, the order of convergence is dropped around discontinuities such as hydraulic jumps and the shoreline. However, the spurious oscillations, specially the oscillations related to the moving shoreline, are bounded and the method is globally mass preserving. The method -Residual Based- has been compared against the Flux Corrected algorithm and the Gradient Jump Viscosity. The Residual Based method has provided better performance for the global situations, steady or transient state, complex topography and reduced artificial viscosity.

The present FIC-FEM procedure has produced accurate results for the examples considered.
In the first example, the artificial diffusion is evaluated and it has been proved to be small and practically inappreciable. The shock capturing term allows to solve supercritical problems with discontinuities and the present procedure is also able to deal with partially wet domains. Finally, a numerical simulation of a dam break flow against an isolated building is performed.
The limitations of the model essentially come from the shallow water equations hypothesis. In fact, the last example presents local regions where the dynamic pressure is not negligible. It is not an obstacle to simulate the main aspects of the flow and the numerical results are in good agreement with the experimental data.

In chapter \ref{eulerian_bsq} the presented stabilization technique has been applied to the Boussinesq equations. Special attention has been paid to higher order derivatives with linear finite elements. Additionally, a numerical approximation to open boundaries has been implemented. The Boussinesq model is able to accurately capture the dispersion effects, which are of crucial importance to analyze the propagation of impulse waves in the context of Landslide Generated Waves (LGW).

Finally, chapter \ref{coupling} presented a novel partitioned strategy for solving landslide-generated wave (LGW) problems. The coupled method makes interact a near-field solver (NFS) with a far-field one (FFS). The NFS reproduces the landslide runout and the impact zone by solving the Navier Stokes equations with the Lagrangian Particle Finite Element Method (PFEM). On the other hand, the FFS uses as input the NFS results stored at a certain interface to model the wave propagation with an Eulerian Finite Element Method (FEM) based on Boussinesq equations. To improve substantially the computational performance of the method and, thus, to allow for the simulation of large-scale problems, we adopt a one-way coupling scheme, meaning that the NFS solution is insensitive to the FFS one. This partitioned method also allows us to freely decouple the time and space discretizations of the two solvers, giving a further advantage in terms of accuracy and efficiency.

The coupling strategy has been analyzed in order to minimize the computational cost, keeping the same accuracy than a fully resolved model. Specifically, the influence of the position of the interface, the temporal domain of the coupling and the open boundary had been analyzed.
The coupled strategy has proven to be accurate and promising results have been obtained. It has been compared against literature an has been applied to a LGW in a representative lake.

All the presented formulations had been implemented in KratosMultiphysics \cite{dadvand2010, dadvand2013}, an open source framework of numerical methods written in C++.


%%%%%%%%%%%%%%%%%%%%%%%%%%%%%%%%%%%%%%%%%%%%%%%%%%%%%%%%%%%%

\section{Further research}


The development of the thesis brings several ideas that can be coped in the future.
Regarding the reduced models for the large scale, some issues can be addressed. The appendix \ref{lagrangian_sw} show that the convergence at the shoreline can be improved. In fact, the stabilized eulerian framework has a first order of convergence, while the lagrangian framework converges at second order. Even though, the eulerian framework is more robust and faster, in global terms it is the preferred option. However, a selective strategy can be developed with the advantages of both frames, for example, an arbitrary lagrangian-eulerian scheme or a cutting technology for partially we elements. Cutting the elements at the shoreline will help to mitigate the Gibbs oscillations since the real free surface does not belong to the finite element space given its discontinuous basis.

A second issue to overcome is the shock capturing for the supercritical regime before the shocks. The non-linear stabilization with the residual viscosity approach exhibits some upstream oscillations, which are nonphysical and related to the numerical model. Tho other non-linear stabilizations are more dissipative and avoid these oscillations, however, the global accuracy is lost due to the excessive diffusion.

Regarding the coupled procedure, the main pending task is the second part of the coupling, from the shallow water equations to the Navier-Stokes equations. Some advances have been made in this regard, but are not fully mature to be included in this thesis as a chapter.

The optimization of the coupling interface in chapter \ref{coupling} has shown that the open boundary is extended over a large domain, thus increasing considerably the computational cost. The development of an open boundary in a Lagrangian frame using the PFEM has been little studied by the scientific community. It will help to achieve higher computational savings up to more than $95\%$. Preliminary analyses suggest that the imposition of an inert paddle moving with the waves can be accurate. However, the generalization to the 3D case with arbitrary directions of propagation is not straightforward.

Finally, a more scientific contribution of the coupled strategy is the extension of the shallow water model with a two way coupled solver. The aim of the two way coupled model is to extend the range of applicability of the reduced model to post breaking waves.

