
\chapter{Conclusions}
\label{conclusions}



%%%%%%%%%%%%%%%%%%%%%%%%%%%%%%%%%%%%%%%%%%%%%%%%%%%%%%%%%%%%

\section{Achievements}




The main objective of this thesis is the investigation of Finite Element formulations applied to large-scale water-related hazards. In the first part of the thesis, a revision of reduced models for the large scale has been done.

In chapter \ref{eulerian_sw} The FIC-FEM procedure has been extended to the shallow water equations. Unlike the FIC-based stabilizations for incompressible flows, the present procedure is applied to the coupled mass and momentum balance at the same time using the linearization matrix $\mathbf{A}_i$. It can be seen that this procedure casts to the classical FIC-stabilization for convection diffusion problems, taking the velocity as linearization term. The same procedure can be applied to develop stabilized formulations for compressible flows.

The present extension of the FIC-procedure to the shallow water equations uses the linearization matrix $\mathbf{A}_i$ for the flux terms to project the characteristic length. However, an alternative framework can be explored with the ASGS \cite{hughes1995,codina2008} formulation, which includes the linearization matrices of the viscous terms and reaction terms. Since the shallow water equations are dominated by the convective matrix $\mathbf{A}_i$, and thus are strictly hyperbolic, the present stabilization is enough to provide stability, as shown in Section \ref{sec:examples}.

The present stabilization provides two algorithmic constants, one for the global stabilization and other one for the shock capturing term. From our numerical experiments, we have chosen $\beta=0.01$ for the stabilization and $\alpha=1.0$ for the shock capturing.

Regarding the accuracy of the shock capturing and the dry domain model, one must notice that this method is not monotonic. Therefore, like in many other stabilized methods, the order of convergence is dropped around discontinuities such as hydraulic jumps and the shoreline. However, the spurious oscillations, specially the oscillations related to the moving shoreline, are bounded and the method is globally mass preserving. The method -Residual Based- has been compared against the Flux Corrected algorithm and the Gradient Jump Viscosity. The Residual Based method has provided better performance for the global situations, steady or transient state, complex topography and reduced artificial viscosity.

The present FIC-FEM procedure has produced accurate results for the examples considered.
In the first example, the artificial diffusion is evaluated and it has been proved to be small and practically inappreciable. The shock capturing term allows to solve supercritical problems with discontinuities and the present procedure is also able to deal with partially wet domains. Finally, a numerical simulation of a dam break flow against an isolated building is performed.
The limitations of the model essentially come from the shallow water equations hypothesis. In fact, the last example presents local regions where the dynamic pressure is not negligible. It is not an obstacle to simulate the main aspects of the flow and the numerical results are in good agreement with the experimental data.

In chapter \ref{eulerian_bsq} the presented stabilization technique has been applied to the Boussinesq equations. Special attention has been paid to higher order derivatives with linear finite elements. Additionally, a numerical approximation to open boundaries has been implemented. The Boussinesq model is able to accurately capture the dispersion effects, which are of crucial importance to analyze the propagation of impulse waves in the context of Landslide Generated Waves (LGW).

Finally, chapter \ref{coupling} presented a novel partitioned strategy for solving landslide-generated wave (LGW) problems. The coupled method makes interact a near-field solver (NFS) with a far-field one (FFS). The NFS reproduces the landslide runout and the impact zone by solving the Navier Stokes equations with the Lagrangian Particle Finite Element Method (PFEM). On the other hand, the FFS uses as input the NFS results stored at a certain interface to model the wave propagation with an Eulerian Finite Element Method (FEM) based on Boussinesq equations. To improve substantially the computational performance of the method and, thus, to allow for the simulation of large-scale problems, we adopt a one-way coupling scheme, meaning that the NFS solution is insensitive to the FFS one. This partitioned method also allows us to freely decouple the time and space discretizations of the two solvers, giving a further advantage in terms of accuracy and efficiency.

The coupling strategy has been analyzed in order to minimize the computational cost keeping the same accuracy resolution than a fully resolved model. Specifically, the influence of the position of the interface, the temporal domain of the coupling and the open boundary had been analyzed.

LAST PARAGRAPH



%%%%%%%%%%%%%%%%%%%%%%%%%%%%%%%%%%%%%%%%%%%%%%%%%%%%%%%%%%%%

\section{Further research}


The development of the thesis brings several ideas that can be coped in the future. 
\begin{enumerate}
    \item ALE/embedded for SW (second order moving boundary)
    \item PFEM absorbing boundaries
    \item Coupling from SW to NS
    \item Impact of two-way as an extended large-scale solver
\end{enumerate}

