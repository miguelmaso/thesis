
\chapter{State of the art}
\label{equations}


\section{Free surface flows}

\subsection{Navier-Stokes equations}


\subsection{Shallow water equations}


\subsection{Boussinesq modified equations}




\section{Numerical methods for the fully resolved model}

\subsection{Overview}

\subsection{Particle methods}



\section{Numerical methods for the reduced models}


\subsection{Overview}

The computation of the dry-wet interface of the shallow water equations is a challenging problem. Due to the hyperbolic character of the equations the water height requires positivity ($h>0$) and oscillations can trigger global failure of the system. Löhner \cite{lohner2008} made a review of possible approximation methods to solve fluid dynamics. Being $u^h = N^i\hat{u}_i$ ($i=1,2,\dots,m$) an approximation of the solution $u$, the weighted residual is defined as

\begin{equation}
\int_{\Omega} W^ir(u^h)d\Omega = 0
\end{equation}

\begin{table}
\centering
\begin{tabular}{|l|c|c|}
\hline
 & $N^i$ & $W^i$ \\ \hline
Finite differences (FD)         & polynomial & $\delta(x_i)$ \\ \hline
Finite volumes (FV)             & polynomial & $1 \ \text{if} \ x\in\Omega_{el}$ \\ \hline
Galerkin finite elements (FEM)  & polynomial & $N^i$ \\ \hline
Discontinuous Galerkin (DG)     & polynomial & ?? \\ \hline
\end{tabular}
\caption{Possible choices of trial and test functions $N^i$ and $W^i$}
\label{possible_trial_functions}
\end{table}

In table \ref{possible_trial_functions} there are a set of possible trial and test functions and the resulting approximation method. In the recent decades a new family of finite elements have been developed which are known as the discontinuous Galerkin (DG). We are interested on the implementation of \emph{continuous Galerkin finite elements} in KratosMultiphysics.


\subsection{Stabilized methods}



\subsection{Monotonicity preserving finite elements}

Following Löhner review, there are three classical approaches: stabilized finite elements \cite{lohner2008}, flux corrected transport (FCT) \cite{lohner2008ch9} and edge based finite elements \cite{lohner2008ch10}.

\paragraph*{Stabilized FEM} Streamline-diffusion methods like SUPG are stable but not
monotonicity preserving. However, Badia and Hierro presented a monotonicity preserving stabilized finite element for hyperbolic equations \cite{badia2014}.

\paragraph*{FCT} The way to circumvent the Godunov barrier theorem \cite{godunov1959} is to develop a nonlinear scheme. FCT uses a low order (LO) monotonic scheme with a lot of diffusion and a high order (HO) oscillatory scheme. The process of combining the two schemes is called limiting:

\begin{equation}
\phi^{n+1} = \phi^n + c_e\Delta \phi_H + (1-c_e)\Delta \phi_L
\end{equation}

\paragraph*{Edge based} The edge based structure is an efficient way to assemble the system matrix which resort to a
finite volume approximation of convective terms. If it is assumed that the fluxes of the variables are constant along the edges, a discontinuity will occur at the edge midpoint. Then, one can replace the Galerkin flux by a Riemann flux and obtain a total variational diminishing scheme (TVD).


\subsection{Particle methods}



\section{Coupling strategies}


