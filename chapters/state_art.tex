
\section{State of the art}
\label{state_art}


In this section a bibliographic revision of the methods employed to solve free surface flows problems is presented.
The general motion of a fluid is described by the Navier-Stokes equations. When free-surface water flows are considered, there are two implications, the incompressibility and homogeneous continuum medium enclosed by the interface. After presenting the most common numerical methods used to approximate the incompressible Navier-Stokes equations for free surface flows, the bibliographic review will be focused on the shallow water equations.
The shallow water equations are derived from the Navier-Stokes equations when some simplifications are assumed.
Finally, the bibliographic revision is closed with the coupling strategies for both models.



\subsection{Numerical methods for the Navier-Stokes equations}


Nowadays, the accurate approximation of the fluid flow equations aims at representing the flow up to the smallest scales. In that sense, the effort is focused either on obtaining higher precision numerical schemes or on refining the discretization. The first approach is cheaper in contrast to the refinement approach, but the refinement approach has been proven to be very versatile. In the case of complex geometries, higher order approximations are equivalent to the \emph{Finite Volume methods} (FV) or \emph{Finite Element Methods} (FEM).

Nevertheless, when the Galerkin discretization is used within the frame of the FEM, an unstable behavior might be obtained. These instabilities are related to the non symmetric convection operator and to interpolation \cite{brezzi1991,codina2008oseen}. The stabilized methods like the \emph{Streamline Upwind Petrov Galerkin} (SUPG) \cite{hughes1986iii,brooks1982} can be framed in the \emph{Variational Multi-Scale} (VMS) concept \cite{hughes1995}. Latter, other stabilizations were presented, such as \emph{Finite-Increment Calculus} (FIC) \cite{onate1998} or \emph{Galerkin Least Squares} (GLS) \cite{hughes1989}. In fact, the latter stabilizations belong to the same family of the previous ones, which consist in adding extra terms based on the residual of the balance equations. Since these stabilization techniques are consistent, they allow using higher order approximations.

However, the terms introduced by stabilizations in order to keep consistency notably increase the complexity of the equations, can couple unknowns, and even increase the non-symmetry of the system. In order to overcome this issues, projection methods only introduce the terms required for stability purpose. The key of these methods is the choice of the projector. A global $L^2$ projector is used in the orthogonal sub scales method \cite{codina2000}. Other methods avoiding the global projection are named local projection stabilization \cite{braack2006,matthies2007} or nodal projection stabilization \cite{badia2012}.


Apart from the stabilization technique for the Navier-Stokes equations, the numerical approximation must be able to deal with the interface discontinuity. Concerning the coordinate frame where the governing equations are solved, the solution methods can be classified in Eulerian and Lagrangian formulations. Classical numerical methods to solve CFD problems typically use the Eulerian formulation with a level set function \cite{chen1997} with enriched shape functions \cite{burman2015cut}. Lagrangian formulations take advantage on the possibility of the FEM to discretize complex geometries. In that case, the FEM is combined with a mesh moving and remeshing strategy. It can be applied in an \emph{Arbitrary Lagrangian-Eulerian formulation} (ALE) \cite{donea2004} or in a pure Lagrangian formulation. This is the case of the \emph{Particle Finite Element Method} (PFEM) \cite{onate2004,idelsohn2004}

Regardless of the framework where the governing equations are solved, the formulation needs to be stabilized in order to fulfill the \emph{inf-sup} condition \cite{brezzi1991}. In this work, the PFEM formulation stabilized with FIC will be used. It has been successfully applied to solve free surface flows \cite{delpin2007} and FSI analysis \cite{onate2008} or in combination with other numerical models \cite{onate2022}.


\subsection{Numerical methods for the shallow water models}


When large scale free surface flows are considered, some simplifications can be assumed, such as small vertical scale and hydrostatic pressure distribution. These assumptions allow to express the Navier-Stokes equations in a depth averaged balance, which are the shallow water equations. Depending on the assumptions, slightly different equations are obtained. While the Saint-Venant equations are well suited for convective flows, the Boussinesq equations describe the oscillatory phenomena of waves. Anyway, both systems of equations are hyperbolic and present an analogy with the Euler conservation laws or compressible Navier-Stokes equations. Hence, the numerical methods commonly used for the compressible Navier-Stokes equations can be applied to the shallow water equations.

The difference of the shallow water equations with respect to the Navier-Stokes equations is the fact that the domain is restricted to the wet area. The wet area is described by the positive water depth, where the depth integration is defined. The shoreline moves according to the dynamics and is characterized by a null water depth, a region where the equations are singular and the system is no longer hyperbolic. This property will need specific numerical considerations.

Löhner \cite{lohner2008} made a review of possible approximation methods to solve fluid dynamics. Being a discretization $u^h = N^i\hat{u}_i$ ($i=1,2,\dots,m$) approximating the solution $u$, the weighted residual is defined as
\begin{equation*}
\int_{\Omega} W^ir(u^h)d\Omega = 0
\end{equation*}
This description allows to wrap the most common numerical methods in the same frame. Depending on the choice of the basis functions $N$ and the trial function $W$, the weighted residual yields several numerical methods. Table \ref{possible_trial_functions} summarizes the most common choices of trial functions.

\begin{table}
\centering
\begin{tabular}{|l|c|c|}
\hline
 & $N^i$ & $W^i$ \\ \hline
Finite differences (FD)         & polynomial & $\delta(x_i)$ \\ \hline
Finite volumes (FV)             & polynomial & $1 \ \text{if} \ x\in\Omega_{el}$ \\ \hline
Galerkin finite elements (FEM)  & polynomial & $N^i$ \\ \hline
Discontinuous Galerkin (DG)     & polynomial & $N^i \ \text{if} \ x\in\Omega_{el}$ \\ \hline
\end{tabular}
\caption{Possible choices of trial and test functions $N^i$ and $W^i$}
\label{possible_trial_functions}
\end{table}

The most commonly used method by the shallow water community is FV because of its stability properties when dealing with the moving shoreline \cite{abbott1978,leveque2002}. More recently, the \emph{Discontinuous Galerkin} technique has been applied to the shallow water equations \cite{ambati2007,khan2014,lee2019}. Like FV, the DG method has the advantage of computing the fluxes at the element boundary, allowing the method to naturally consider the moving shoreline. However, the introduction of high order DG schemes is not straightforward.

In this research the FEM for the shallow water equations will be explored. Apart from its ability to solve the oscillatory problem of the Boussinesq equations, it will allow a unified formulation for the coupling strategy. When the FEM is considered, several sources of instabilities arise. Firstly, the \emph{inf-sup} condition if the same space of interpolation is used for all the variables. Secondly, analogously to compressible flows, shocks might appear in the regime variations, from sub-critical to super-critical. Lastly, Gibbs oscillations appear at the shoreline, since the partially wet elements are not able to preserve the water positivity.

Several authors reported the presence of instabilities on the numerical approximation of the shallow water equations on the wet domain. Ad-hoc techniques were proposed in order to achieve stability, such as a DG for one variable \cite{gourgue2009}, different order of interpolation \cite{leclerc1990}, division in sub-triangles \cite{heniche2000}, non conforming interpolations \cite{hanert2005} or addition of extra diffusion \cite{defina2000}. A similar situation happened with the Boussinesq equations, see \cite{walkley2002,woo2004a} as an example. Heniche related the instabilities to the \emph{inf-sup} condition in \cite{heniche2000}, but until Codina in \cite{codina2008} the FEM stabilization was not applied to the shallow water equations.

The stabilized FEM are known to ensure global stabilization, however, monotonic properties are not inherited by the numerical scheme. This issue is related to the dry-wet interface, since the wet domain is defined by a positive water depth. Usually, the front tracking is related to the techniques used for the shock capturing, since the shoreline is also a discontinuity.
Monotonic properties of the numerical approximation are related to the Godunov barrier theorem \cite{godunov1959} which states that a scheme of order higher than 1 is not oscillatory free. The \emph{Flux Corrected Transport} FCT algorithm \cite{lohner2008ch9} uses the Godunov theorem to construct a non linear scheme which combines a low order non-oscillatory scheme with a high order one. The process of combining the two schemes is called limiting. An application of it to the shallow water equations can be found in \cite{ortiz2012}.

Other methods inspired by the flux limiting are the edge based schemes \cite{lohner2008ch10}. The edge based structure is an efficient way to assemble the system matrix which resort to a FV approximation of convective terms. If it is assumed that the fluxes of the variables are constant along the edges, a discontinuity will occur at the edge midpoint. Then, one can replace the Galerkin flux by a Riemann flux and obtain a \emph{Total Variational Diminishing} scheme (TVD).

Finally, non-linear stabilizations are a more consistent way to add local stabilizations around shocks and the moving shoreline \cite{codina2011}. Badia and Hierro presented a monotonicity preserving stabilized finite element for hyperbolic equations \cite{badia2014}.



\subsection{Large-local scale approach and coupling strategies}


The simulation of a large scenario where local effects are of key importance can be tackled in different ways. In many practical situations, the behavior of a fluid can be modelled by the SW equations, even though the assumptions of small vertical acceleration are not perfectly fulfilled. However, the accurate modelling of the event triggering the impulse wave or the interaction with structure can be out of the scope of reduced models.

On the one hand, physical models are particularly helpful to identify the key parameters of the flow characteristics. The authors in \cite{noda1970water, fritz2004near, mulligan2017} experimentally characterized the parameters for the definition of the impulse wave generated by a landslide. \cite{krautwald2020,krautwald2022} performed an extensive characterization of the wave run-up and interaction with a nearshore structure. Nevertheless, laboratory tests are mainly devoted to determine near-field conditions under specific simplifications, whereas the global scenario is a combination of a variety of situations.

On the other hand, numerical methods have the potential to predict both near- and far-field waves characteristics. Due to its computational requirements when large domains and long time durations are considered, numerical methods can be classified into three main groups.

Firstly, the entire domain can be simulated using a reduced model, typically based on SW. For the case of LGW, the impact can be imposed as an equivalent boundary condition \cite{waythomas2003numerical, ataie2008mapping}. In the case of the interactions with bodies, some large scale applications have been modelled using the SW equations for FSI problems \cite{geveler2010,bresch2021}. However, the strong simplifications allow to have only an approximate idea of the global scenario.

Secondly, the global scenario can be simulated in a unique coupled model. Some holistic attempts have been made both for the wave generation and the impact against structures (see \cite{CrostaVajont,franci20203dA,zhu2018} for example). However, the computational cost for a fully resolved method is still unaffordable for large-scale events.

Lastly, the partitioned approach splits the global scenario into several simulations that interact with each other at their interface. A first inspiration of the strong coupling of different order models can be found in \cite{formaggia2001}. There are also some recent strong couplings between the SW and Navier-Stokes equations \cite{pringle2016}. However, the weak coupling simplification preserves the computational advantages of the partitioned approach and it still ensures an accurate modeling of the key phenomena of an wave generation scenario or a FSI scenario.

One of the first applications of the weakly-coupled partitioned method for LGWs was presented in \cite{heinrich1998simulation}. In this work, a simplified 3D model was used for the landslide-water impact and a shallow water model was applied for the far-field wave propagation. Other examples can be found in \cite{lovholt2008oceanic} where a flooding scenario was studied by coupling a 3D compressible Eulerian solver with a Boussinesq model.

In this work, we propose and validate a novel partitioned model for LGW scenarios. In this new strategy, a Lagrangian finite element method, namely the Particle Finite Element Method (PFEM) \cite{idelsohn2004, onate2004, cremonesi2020state}, is used as the NFS and a Boussinesq model is used as FFS. Particular attention is devoted here to analyze the effect of the near-field boundary conditions on the far-field propagating wave.


