
\section{State of the art}
\label{state_art}


In this section a bibliographic revision of the methods employed to solve free surface flows problems is presented.
The general motion of a fluid is described by the Navier-Stokes equations. When free-surface water flows are considered, there are two implications, the incompressibility and homogeneous continuum medium enclosed by the interface. After presenting the most common numerical methods used to approximate the incompressible Navier-Stokes equations for free surface flows, the bibliographic review will be focused on the shallow water equations.
The shallow water equations are derived from the Navier-Stokes equations when some simplifications are assumed.
Finally, the bibliographic revision is closed with the coupling strategies for both models.



\subsection{Numerical methods for the Navier-Stokes model}


Nowadays, the accurate approximation of the fluid flow equations is aimed to represent the flow up to the smallest scales. In that sense, the effort is focused either on obtaining higher precision numerical schemes or in refining the discretization. The first approach is cheaper in contrast to the refinement approach, but the refinement approach have been proven to be much versatile. Even more, it is usually restrained to simple domain geometries. In the case of complex geometries, higher order approximations are equivalent to the \emph{Finite Volume methods} (FV) or \emph{Finite Element Methods} (FEM).

Nevertheless, when the Galerkin discretization is used within the frame of the FEM, an unstable behavior might appear. These instabilities are related to the non symmetric convection operator and to interpolation \cite{brezzi1991,codina2008oseen}. The stabilized methods like the \emph{Streamline Upwind Petrov Galerkin} (SUPG) \cite{hughes1986iii,brooks1982} can be framed in the \emph{Variational Multi-Scale} (VMS) concept \cite{hughes1995}. Latter, other stabilizations were presented, such as \emph{Finite-Increment Calculus} (FIC) \cite{onate1998} or \emph{Galerkin Least Squares} (GLS) \cite{hughes1989}. In fact, those stabilizations are of the same family of the previous ones, which consist on adding extra terms based on the residual of the balance equations. Since these stabilization techniques are consistent, allow for using higher order approximations.

However, the number of terms introduced by stabilizations in order to keep consistency notably increase, can couple unknowns and increase the non-symmetry of the system. In order to overcome this issues, projection methods only introduce the terms required for stability purpose. The key of these methods is the choice of the projector. A global $L^2$ projector is used in the orthogonal sub scales method \cite{codina2000}. Other methods avoiding the global projection are named local projection stabilization \cite{braack2006,matthies2007} or nodal projection stabilization \cite{badia2012}.


Apart from the stabilization technique for the Navier-Stokes equations, the numerical approximation must be able to deal with the interface discontinuity. Concerning the coordinate frame where the governing equations are solved, the solution methods can be classified in Eulerian and Lagrangian formulations. Classical numerical methods to solve CFD problems typically use the Eulerian formulation with a level set function \cite{chen1997} with enriched shape functions \cite{burman2015cut}. Very often, Lagrangian formulations take advantage on the possibility of the FEM to discretize complex geometries. In that case, the FEM is combined with a mesh moving and remeshing strategy. It can be applied in an \emph{Arbitrary Lagrangian-Eulerian formulation} (ALE) \cite{donea2004} or in a pure Lagrangian formulation. This is the case of the \emph{Particle Finite Element Method} (PFEM) \cite{onate2004,idelsohn2004}

Regardless the framework where the governing equations are solved, the formulation needs to be stabilized in order to fulfill the \emph{inf-sup} condition \cite{brezzi1991}. In this work, the PFEM formulation stabilized with FIC will be used. It has been successfully applied to solve free surface flows \cite{delpin2007} and FSI analysis \cite{onate2008} or in combination with other numerical models \cite{onate2022}.


\subsection{Numerical methods for the shallow water models}


\subsubsection{Overview}

The computation of the dry-wet interface of the shallow water equations is a challenging problem. Due to the hyperbolic character of the equations the water height requires positivity ($h>0$) and oscillations can trigger global failure of the system. Löhner \cite{lohner2008} made a review of possible approximation methods to solve fluid dynamics. Being $u^h = N^i\hat{u}_i$ ($i=1,2,\dots,m$) an approximation of the solution $u$, the weighted residual is defined as

\begin{equation}
\int_{\Omega} W^ir(u^h)d\Omega = 0
\end{equation}

\begin{table}
\centering
\begin{tabular}{|l|c|c|}
\hline
 & $N^i$ & $W^i$ \\ \hline
Finite differences (FD)         & polynomial & $\delta(x_i)$ \\ \hline
Finite volumes (FV)             & polynomial & $1 \ \text{if} \ x\in\Omega_{el}$ \\ \hline
Galerkin finite elements (FEM)  & polynomial & $N^i$ \\ \hline
Discontinuous Galerkin (DG)     & polynomial & $N^i \ \text{if} \ x\in\Omega_{el}$ \\ \hline
\end{tabular}
\caption{Possible choices of trial and test functions $N^i$ and $W^i$}
\label{possible_trial_functions}
\end{table}

In table \ref{possible_trial_functions} there are a set of possible trial and test functions and the resulting approximation method. In the recent decades a new family of finite elements have been developed which are known as the discontinuous Galerkin (DG). We are interested on the implementation of \emph{continuous Galerkin finite elements} in KratosMultiphysics.


\subsubsection{Stabilized methods}



\subsubsection{Monotonicity preserving finite elements}

Following Löhner review, there are three classical approaches: stabilized finite elements \cite{lohner2008}, flux corrected transport (FCT) \cite{lohner2008ch9} and edge based finite elements \cite{lohner2008ch10}.

\paragraph*{Stabilized FEM} Streamline-diffusion methods like SUPG are stable but not
monotonicity preserving. However, Badia and Hierro presented a monotonicity preserving stabilized finite element for hyperbolic equations \cite{badia2014}.

\paragraph*{FCT} The way to circumvent the Godunov barrier theorem \cite{godunov1959} is to develop a nonlinear scheme. FCT uses a low order (LO) monotonic scheme with a lot of diffusion and a high order (HO) oscillatory scheme. The process of combining the two schemes is called limiting:

\begin{equation}
\phi^{n+1} = \phi^n + c_e\Delta \phi_H + (1-c_e)\Delta \phi_L
\end{equation}

\paragraph*{Edge based} The edge based structure is an efficient way to assemble the system matrix which resort to a
finite volume approximation of convective terms. If it is assumed that the fluxes of the variables are constant along the edges, a discontinuity will occur at the edge midpoint. Then, one can replace the Galerkin flux by a Riemann flux and obtain a total variational diminishing scheme (TVD).


% \subsubsection{Particle methods}



\subsection{Coupling strategies}


Accurate modeling and prediction of LGWs are of key importance to reduce their catastrophic effects.
Both experimental and numerical studies have been greatly contributed to the enhancement of the forecasting capabilities against these natural hazards.

Physical models are particularly helpful to identify the key parameters of both the sliding material and the water body and to determine their specific effect on the LGW scenario \cite{noda1970water, fritz2004near, heller2011wave, mohammed2012physical, romano2016, mulligan2017, evers2019spatial}. A detailed overview of experimental tests applied to LGW events can be found in \cite{yavari2017subaerial}. Nevertheless, laboratory tests are mainly devoted to determining the near-field wave conditions, while estimations on the far-field waves, which are responsible for major damages to the coastal areas affected by an LGW event, are more difficult to extrapolate.

On the other hand, numerical methods have the potential to predict both near- and far-field waves characteristics. However, the numerical simulation of a LGW is a challenging task. Indeed, the computational method must be able to model the complex constitutive behavior of the landslide material, deal with fluid-solid (or multi-fluid) interaction, and track the largely evolving topology of both landslide and water bodies. Furthermore, the LGW analysis involves different characteristic time and space scales for the near field (landslide-water impact zone) and the far field (wave propagation). Finally, it is required to solve large-scale three-dimensional (3D) geometries for long time durations, and this makes the computational cost of LGW analyses one of the most critical issues.

The numerical models applied to LGWs can be classified into three main groups \cite{yavari2016numerical} briefly summarized below.

The first approach consists in using a wave propagation solver, typically based on Shallow Water (SW) equations. In this strategy, the landslide runout and water impact are not resolved but are introduced into the model as an equivalent boundary condition \cite{waythomas2003numerical, ataie2008mapping}. This approach is the simplest one and has the lowest computational cost. However, it assumes strong simplifications on both the landslide motion and momentum transfer and thus it can only give an approximate idea of the global LGW scenario.

In the second strategy, the landslide and water motion equations are solved in a unique coupled model. First applications of this holistic strategy can be found in \cite{kelfoun2010landslide} and \cite{giachetti2011numerical}, where shallow water models were used for both the landslide and the water body.
Only recently, more accurate 3D monolithic approaches for LGW problems have been presented, see for example
\cite{vacondio20133d, CrostaVajont, franci20203dA, franci20203dB, xu2021sph, terada2021}. Nevertheless, the computational cost of this fully resolved method can be still unaffordable for large-scale events.

The third approach splits the LGW problem into two simulations that interact with each other at their interface. Typically, in these so-called partitioned strategies, a numerical method, here called $near$-$field$ $solver$ (NFS), computes the landslide runout, impact against the water body and wave formation. A different numerical scheme, here called $far$-$field$ $solver$ (FFS), predicts the far-field wave propagation \cite{yavari2016numerical}.
Generally, a weak (or one-way) coupling scheme is adopted, meaning that the NFS is insensitive to the FFS solution. The one-way coupling simplification preserves the computational advantages of this partitioned approach and it still ensures an accurate modeling of the key phenomena of an LGW scenario, such as the landslide runout, the wave generation and the far-field wave propagation.
%In order to keep the computational advantages of this partitioned approach, a weak (or one-way) coupling scheme is generally adopted, meaning that the NFS is insensitive to the FFS solution. The one-way coupling simplification has a negligible effect both on the water wave generation and on the far-field wave propagation and runup, which are the key phenomena of an LGW scenario.
 
One of the first applications of this partitioned method for LGWs was presented in \cite{heinrich1998simulation}. In this work, a simplified 3D model was used for the landslide-water impact and a shallow water model was applied for the far-field wave propagation.
In \cite{lovholt2008oceanic}, a potential tsunami scenario induced by the collapse of a part of Cumbre Vieja Volcano of La Palma island, Spain, was studied by coupling a 3D compressible Eulerian solver with a Boussinesq model. In \cite{abadie2012numerical}, the same case study was analyzed using a 3D Volume Of Fluid (VOF) method, as the NFS, and an analogous FFS such as that used in \cite{lovholt2008oceanic}.
More recently, Tan et al. \cite{tan2018numerical} coupled a Smoothed Particle Hydrodynamics (SPH) method with a shallow water equations solver was used to reproduce hypothetical LGW scenarios at Es Vedrà, Ibiza, Spain.

%In this work, we propose and validate a novel partitioned model for LGWs. In this new strategy, a Lagrangian finite element method, namely the Particle Finite Element Method (PFEM) \cite{PFEM2004, onate2004, cremonesi2020state}, is used as the NFS and a standard shallow water Boussinesq model is used as the FFS. Several previous works have shown the accuracy of the PFEM to model landslides \cite{PFEMzhang1, PFEMzhangLandslide, zhang2021gpu}, also in cascading events \cite{CremonesiPFEM2, Salazar2012, CremonesiSlipLandslides, ZhangSubmarine}. In this work, we use the PFEM approach that has been successfully applied to LGW scenarios in \cite{mulligan2020} and in \cite{franci20203dA, franci20203dB}, where 3D simulations of the Vajont disaster were presented. 
%This work aims at being a proof of concept of this new coupled strategy for real LGW scenarios. For this reason, a deep validation of the method is presented by analyzing the performance and accuracy of the new partitioned technique in targeted tests, using reference solutions obtained with other numerical methods, experimental tests and analytical solutions.
%In partitioned methods, the momentum transfer between the Navier-Stokes and the Boussinesq models must be accurate in order to obtain a faithful representation of the LGW scenario. Thus, particular attention is devoted here to analyze the effect of the near-field boundary conditions on the far-field propagating wave. Convergence and sensitivity analyses are carried out in order to verify the accuracy and robustness of the proposed method. 



