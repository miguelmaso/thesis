
\chapter{Finite Element Methods for the Boussinesq modified equations}
\label{eulerian_bsq}


In this chapter, a FEM approximation for the Boussinesq modified equations is presented. As stated in section \ref{equations}, the Boussinesq equations and the Saint Venant equations are of the same family. Apart from the choice of different primary variables, the main difference consists on the inclusion of the dispersive terms. However, part of the structure of the equations remains unmodified.

The weak principle, the linear stabilization and the spatial discretization presented in section \ref{eulerian_sw} will be directly applied for the Boussinesq modified equations. The main differences arise from the dispersion terms.
First of all, the dispersion terms include spatial derivatives of order higher than two. Thus, special techniques for considering the third order derivatives will be included in the FEM procedure.

Secondly, the time discretization will be modified using a semi-implicit fourth-order scheme. The new time scheme allows to deal with the oscillatory behavior of the Boussinesq equations.
Additionally, the matrix structure will suffer some modifications related to the high order derivatives and the time integration scheme.

Finally, the shock capturing is not needed, since the dispersion effect prevent from breaking and the formation of steep gradients. However, in the vicinity of the shoreline or very shallow domains, the amplitude dispersion effect could dominate over the frequency dispersion, making very recommended the inclusion of a shock capturing technique. Furthermore, if the shoreline is included, the shock capturing is required, since it provides stabilization for the moving front.

On another note, a section regarding the numerical treatment of absorbing boundary conditions is included. The inclusion of the frequency dispersion and the need of shortening the computational domain, lead to the open boundary conditions. On an open boundary conditions, waves can exit the computational domain and reflections must be avoided. The numerical approximation for the open boundary conditions are called absorbing boundary conditions.

Finally, the chapter is closed with some examples. The examples are designed in order to test the accuracy and to show the capabilities of the presented algorithms.




\section{Stabilized formulation for the Boussinesq modified equations}


The Boussinesq modified equations need to be stabilized. In this section the linear stabilization presented in \ref{sec:fic_fem_stabilization} will be applied. Since the Boussinesq equations are intended to solve oscillatory problems, the shock capturing is not relevant. However, an additional technique must be introduced in order to deal with the higher order derivatives.

In this section, the boundary conditions considered by the numerical model will be introduced. Then, the numerical procedure is explained. Here, the introduction of a gradient recovery is presented. After presenting the formulation for solving the equations, a numerical procedure for the absorbing boundary conditions will be proposed. Finally, some examples are added.



\subsection{Boundary conditions}


There are three types of boundary conditions considered for the Boussinesq problems, inflow $\Gamma_I$, reflecting $\Gamma_R$ and absorbing $\Gamma_A$ boundaries. Those subdomains are such that $\Gamma_I \cup \Gamma_R \cup \Gamma_A = \partial \Omega$, being $\partial\Omega$ the boundary of the domain.

\paragraph{Inflow boundary $\Gamma_I$} Both free surface and velocity are known at the boundary. Typically it is used to impose a wave generator. Since the wave amplitude is known, the horizontal velocity can be obtained using linear wave theory.

\paragraph{Reflecting boundary $\Gamma_R$} No fluid should pass through an impermeable wall. This implies imposing the normal component of the velocity to be zero.
\begin{equation*}
    \bar{\mathbf{u}} \cdot \mathbf{n} = 0 \quad \text{on} \ \Gamma_R
\end{equation*}
Following Woo and Liu \cite{woo2004a}, the above relation must be rewritten in terms of $\mathbf{u}_\beta$ and the velocities are related as
\begin{equation*}
    \bar{\mathbf{u}} = \mathbf{u}_\beta + H^{-1} \mathbf{J}_\eta
\end{equation*}
Hence, the complete formulation of a reflective boundary is
\begin{equation}
    \bar{\mathbf{u}}_\beta \cdot \mathbf{n} = 0 \quad
    \mathbf{J}_\eta \cdot \mathbf{n} = 0 \quad
    \text{on} \ \Gamma_R
\end{equation}

\paragraph{Absorbing boundary $\Gamma_A$} An outgoing wave should not return to the computational domain. A practical implementation of the absorbing boundaries are the sponge layers \cite{israeli1981, wei1995}, where som extra diffusion and a newtonian damper are added near the boundary. Those artificial coefficients varies exponentially, from zero to a given value near the boundary.



\subsection{Finite Element formulation}

\subsubsection{Spatial discretization}

The Boussinesq equations are solved using the FEM. The space domain is interpolated with a Galerkin discretization of linear triangles and a finite difference scheme with constant time step is used to integrate the equations in time.
Some numerical difficulties such as the third order differential operator and the time integration accuracy are addressed in \cite{walkley2002,woo2004a,wei1995}.
As stated by Codina in \cite{codina2008,codina2008b} the problem (\ref{bsq_eq}) is an hyperbolic wave in mixed form and there is an incompatibility condition (see \cite{BrezziFortin}) because the same interpolation is used for both variables, the velocity $\mathbf{u}_\beta$ and the wave amplitude $\eta$.
Here the equations are stabilized using FIC extending the work done in \cite{maso2022}.


\subsubsection{Derivatives recovery}

Following \cite{walkley2002}, the third order spatial derivatives are modelled using $\mathbf{J}_\eta$ as an intermediate variable. Since it can be solved in a staggered way, the number of degrees of freedom is not increased and thus the performance of the scheme is unaltered. The auxiliary field $\mathbf{J}_\eta$ is computed using the Zhang \cite{zhang2005} super-convergent gradient recovery.
Finally, the problem (\ref{bsq_eq}) is expressed in matrix form as
\begin{equation} \label{sw_system_matrix}
    (M + K) \dot{x} = F(x,y)
\end{equation} 
where $x$ is the vector of nodal unknowns, $y$ is the vector of nodal values corresponding to $\mathbf{J}_\eta$, $M$ is a mass matrix, $K$ correspond to the second derivatives associated to $\mathbf{J}_{\mathbf{u}}$, and $F$ is a non-linear vector which is a function of $\mathbf{u}_\beta$ and its spatial derivatives.


\subsubsection{Time discretization}

The commonly used fourth order Adams-Moulton scheme is employed for the time integration of (\ref{sw_system_matrix}), see \cite{wei1995,woo2004a,codina2008b} as an example.
To obtain a solution at the time step $t^{n+1}$, the fixed point iterative method is employed to deal with the non-linearity of the system. Given a guess $x^{n+1,i-1}$ for $x^{n+1,i}$ at iteration $i$, the increment $\delta x^i$ is computed from
\begin{multline}
    24 (M + K) \delta x^i = 
    24 (M + K) (x^n - x^{n+1,i-1}) \\
     + \delta t (9F^{n+1} + 19F^n - 5F^{n-1} + F^{n-2}) + O(\delta t^4)
    \label{adams-moulton}
\end{multline}
The time integration is closed using the explicit third order Adams-Bashforth scheme to predict the initial guess $x^{n+1,0}$ for the non-linear iterations
\begin{multline}
    12 (M + K) x^{n+1,0} = 
    12 (M + K) x^n \\
     + \delta t (23F^n - 16F^{n-1} + 5F^{n-2}) + O(\delta t^3)
    \label{adams-bashforth}
\end{multline}

The system of (\ref{adams-moulton}) and (\ref{adams-bashforth}) is solved using a direct solver. Given the problem is small, this is not expensive.
Convergence is achieved when $\delta x^i$ is sufficiently small. Then, the solution $x^{n+1}$ is set as $x^{n+1,i}$.





\subsection{Absorbing boundary conditions}

Frequently, the need to shorten the numerical domain arises. This can be achieved by the imposition of open boundaries, also known as radiant boundaries. The open boundaries allow the exit of the waves as well as the consistency of the system of equations in order to ensure the existence and uniqueness of the solution. Additionally, the numerical tool for the boundary has to be compatible with the numerical approach for the inner domain.

Generally, the radiant boundary condition denotes the analytical formulation for open boundaries and the term absorbing boundary is related to the numerical approximation of the radiation condition \cite{navon2004}. An equilibrium between the precision offered by the radiant boundary and the numerical cost required by the absorbing boundary is seek.
From one side, the boundary conditions must define a well posed problem and the spurious oscillations in the open boundary shall be as small as possible. From the other side, the computational cost of the boundary conditions should be small compared to the computational cost of the inner domain.

Considering the unidirectional wave propagation and being $\eta$ the free surface elevation, a wave function will verify a radiant boundary condition where
\begin{equation} \label{radiation_condition}
    \pder{\eta}{t} + c \cos(\theta)\pder{\eta}{x} = 0
\end{equation}
where $\theta$ is tha incidence angle of the wave with respect to the boundary.

Some authors proposed local approximations highly diffusives \cite{engquist1977}. Latter, Bayliss explored the radiation boundary conditions for the Helmholtz equations in \cite{bayliss1982}.
Similarly, Collino \cite{collino1993} extended the formulation using higher order of derivatives than \ref{radiation_condition}. However, the generalization for the Euler equations -analogously the SW equations- in 2 or 3D is not trivial, since the incidence angle $\theta$ is not known \emph{a priori} and because of the dispersive behavior of the equations \cite{wei1995}.

The dispersive behavior is associated to the fact that there is not an unique celerity characterizing the system: in the case of the SW or the Boussinesq equations there are three waves superposed propagating at different speeds. For example, in section \ref{sec:sw_fc} a decomposition in characteristics was presented through diagonalization of the tangent matrices $\mathbf{A}_i$.
\begin{equation}
    \pder{\bm\Phi_j}{t} + \bm\Lambda_i \pder{\bm\Phi_j}{x_i} = 0
\end{equation}
in that minimal wave equation $\bm\Lambda$ is a diagonal matrix such that $\bm A_i = \bm T_i \bm\Lambda_i \bm T_i^{-1}$.
This problem can be understood a the superposition of three waves, each one of them propagating at speed $u-c$, $u$ y $u+c$, namely the eigenvalues $\lambda_i$. The sign of $\lambda_i$ determines if the wave is incoming or outgoing and the regime of the problem: subcritical or supercritical.

Besides having to impose the radiant boundary condition for the three waves, this approximation is limited, since in 2 or 3D does not exist a genuine characteristic-based formulation. A more detailed development can be found in \cite{lie2001}.

Most of the proposals to approximate the open boundaries consist on relaxing the condition \ref{radiation_condition}  adding a dissipation term $K$ and extending the domain a distance $d$, losing the local definition. See \cite{israeli1981,navon2004,carmigniani2018} as an example. Then, the radiation condition is rewritten as
\begin{equation} \label{radiation_absorbing_condition}
    \pder{\eta}{t} + c \cos(\theta)\pder{\eta}{x} = -K\eta
\end{equation}
This approximation can be used in combination with local absorbing boundary conditions \cite{wei1995}.

Another of the explored possibilities is the Perfectly Matched Layer (PML) \cite{berenger1994}, widely used in the literature. It presents the inconvenient of special treatment of corners.
Later, this methodology wa used in combination with the high order absorbing boundaries, leading to the Double Absorbing Boundary (DAB), proposed by Hagstrom and Rabinovich in \cite{hagstrom2014,rabinovich2015}. This technology places two parallel boundaries separated by a short distance. Its advantage is that they do not need to incorporate the derivatives in the normal direction neither to apply any special treatment to the corners.

Finally, and given to the simplicity of the formulation, there is the possibility of including only a sponge layer. It can be found in the formulations of Israeli and Carmigniani in \cite{israeli1981,carmigniani2018}. A Newtonian dissipative term analogous to the bottom friction $S_f$ is added as
\begin{equation}
    S_a = -\gamma(\mathbf{x}) \mathbf{u}
\end{equation}
The dissipative term $\gamma$ varies from 0 at a given distance to the discrete boundary to the maximum value $\gamma_{max}$ at the discrete boundary. This variation follows an exponential law \cite{peric2018} of order $n$. The parameters of the sponge layer $d_0$ and the maximum value $\gamma_{max}$ need to be specified for each case, in terms of the wavelength.
A possible expression for $\gamma$ is
\begin{equation} \label{exponential_sponge_layer}
    \gamma(\mathbf{x}) = \gamma_{\max} \mathcal{H}(d_0 - d(\mathbf{x})) \frac{e^{\left(\frac{d_0 - d(\mathbf{x})}{d_0}\right)^n} - 1}{e - 1}
\end{equation}
where $\mathcal{H}$ is the Heaviside function, $d$ is the distance from a point to the computational boundary.

In Fig. \ref{absorbing_boundary} the propagation of a train of regular waves is shown. Ate the left end of the domain, waves are generated. At the right end of the domain there is a sponge layer. Studies carried out by Carmigniani in \cite{carmigniani2018} show that some combinations of the maximum absorption, layer width and exponential degree can produce undesired reflections. For example, an excessive absorption can generate reflections at the beginning of the sponge layer.

An exponent $n=3$ for expression \ref{exponential_sponge_layer} is chosen, it considerably minimizes the reflection \cite{carmigniani2018}. A sensibility analysis of the parameters $d_0$ y $\gamma_{\max}$ should be done.
It is useful to express the maximum distance and absorption in terms of the wavelength and frequency. In practice, the width of the sponge layer is used to be $d_0 \approx 2\lambda$ and the absorption coefficient $\gamma \approx 0.5\omega$.

The reflected amplitude is monotonically decreasing as the sponge width increases. Nevertheless, the reflection exhibits a local minimum with respect to the absorption coefficient. This makes highly recommended to adjust the parameters of the sponge layer according to the  maximum wavelength.


\begin{figure}
    \centering
    \includegraphics[width=.8\textwidth]{img/absorbing_boundary/absorbing_boundary.pdf}
    \caption{Propagation and absorption of a train of waves. The shadow region shows the width of the sponge layer.}
    \label{absorbing_boundary}
\end{figure}


\subsection{Examples}



\section{Concluding remarks}


