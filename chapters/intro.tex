
\chapter{Introduction}
\label{chapter_introduction}

Floods are complex natural phenomena with a great assortment of causes and a huge casuistic
depending on the land characteristics or the climatic conditions. Preventing floods is still a great
challenge due to uncertainty of climatic conditions and complexity of modeling land, usually
defined by a great number of variables and parameters.
Some of the main flooding causes are torrential storms, hurricanes or monsoon, reservoir failure
or small floodplains. Near coastal zones floods are also caused by storm surges or even
tsunami waves. In the case of lakes and reservoirs, long waves may be generated by landslides.

Flooding is the most important natural hazard in terms causing of loss of life, displacement of
people and economic losses. But vulnerability can
increase when natural hazards make strategic structures fail, or when there aren't precautionary
measures or actuation plans.



Even thoug this study is about the numercial modelling of water related hazards and special attention has been paid to hydrological problems, the equations can be applied to a huge variety of problems. The configuration is modifyed just specifying the appropriate boundary conditions. Aside from the main scope of the thesis, some other examples are included. One example of application is related to pollutant transport in atmospheric environment. The other example is about tune liquid dampers for high rise buildings.



\section{Objectives} 

The main objective of this thesis is the coupling of different scales in natural hazards. Usually, the action is originated in a far filed from where the the vulnerability is located. There are three scenarios with different scale. The action may be generated in a local or global scale, depending on the nature of the phenomena. The propagation is usually in the large scale and it introduces the need of using reduced models. Finally, the evaluation of the resilience of singular structures requires to descend to the local scale.

The objectives can be sumarized as
\begin{itemize}
    \item aaa
\end{itemize}

Gran escala: simplificaciones 2D a las ecuaciones de Navier-Stokes.

Escala local: fenómenos turbulentos que requieren la solución de las ecuaciones de Navier-Stokes en 3D con alta resolución.


\section{State of the art}


[DESARROLLAR ESTADO DEL ARTE]

The computation of the dry-wet interface of the shallow water equations is a challenging problem. Due to the hyperbolic character of the equations the water height requires positivity ($h>0$) and oscillations can trigger global failure of the system. Löhner \cite{lohner2008} made a review of possible approximation methods to solve fluid dynamics. Being $u^h = N^i\hat{u}_i$ ($i=1,2,\dots,m$) an approximation of the solution $u$, the weighted residual is defined as

\begin{equation}
\int_{\Omega} W^ir(u^h)d\Omega = 0
\end{equation}

\begin{table}
\centering
\begin{tabular}{|l|c|c|}
\hline
 & $N^i$ & $W^i$ \\ \hline
Finite differences (FD)         & polynomial & $\delta(x_i)$ \\ \hline
Finite volumes (FV)             & polynomial & $1 \ \text{if} \ x\in\Omega_{el}$ \\ \hline
Galerkin finite elements (FEM)  & polynomial & $N^i$ \\ \hline
Discontinuous Galerkin (DG)     & polynomial & ?? \\ \hline
\end{tabular}
\caption{Possible choices of trial and test functions $N^i$ and $W^i$}
\label{possible_trial_functions}
\end{table}

In table \ref{possible_trial_functions} there are a set of possible trial and test functions and the resulting approximation method. In the recent decades a new family of finite elements have been developed which are known as the discontinuous Galerkin (DG). We are interested on the implementation of \emph{continuous Galerkin finite elements} in KratosMultiphysics.



% \subsection{Monotonicity preserving finite elements}

Following Löhner review, there are three classical approaches: stabilized finite elements \cite{lohner2008}, flux corrected transport (FCT) \cite{lohner2008ch9} and edge based finite elements \cite{lohner2008ch10}.

\paragraph*{Stabilized FEM} Stramline-diffusion methods like SUPG are stable but not
monotonicity-preserving. However, Badia and Hierro presented a monotonicity preserving stabilized finite element for hyperbolic equations \cite{badia2014}.

\paragraph*{FCT} The way to circumvent the Godunov barrier theorem \cite{godunov1959} is to develop a nonlinear scheme. FCT uses a low order (LO) monotonic scheme with a lot of diffusion and a high order (HO) oscillatory scheme. The process of combining the two schemes is called limiting:

\begin{equation}
\phi^{n+1} = \phi^n + c_e\Delta \phi_H + (1-c_e)\Delta \phi_L
\end{equation}

\paragraph*{Edge based} The edge based structure is an efficient way to assemble the system matrix which resort to a
finite volume approximation of convective terms. If it is assumed that the fluxes of the variables are constant along the edges, a discontinuity will occur at the edge midpoint. Then, one can replace the Galerkin flux by a Riemann flux and obtain a total variational diminishing scheme (TVD).


