
\chapter{Introduction}
\label{chapter_introduction}

Floods are complex natural phenomena with a great assortment of causes and a huge casuistic
depending on the land characteristics or the climatic conditions. Preventing floods is still a great
challenge due to uncertainty of climatic conditions and complexity of modeling land, usually
defined by a great number of variables and parameters.
Some of the main flooding causes are torrential storms, hurricanes or monsoon, reservoir failure
or small floodplains. Near coastal zones floods are also caused by storm surges or even
tsunami waves. In the case of lakes and reservoirs, long waves may be generated by landslides.

Flooding is the most important natural hazard in terms causing of loss of life, displacement of
people and economic losses. But vulnerability can
increase when natural hazards make strategic structures fail, or when there aren't precautionary
measures or actuation plans.



Even thoug this study is about the numercial modelling of water related hazards and special attention has been paid to hydrological problems, the equations can be applied to a huge variety of problems. The configuration is modifyed just specifying the appropriate boundary conditions. Aside from the main scope of the thesis, some other examples are included. One example of application is related to pollutant transport in atmospheric environment. The other example is about tune liquid dampers for high rise buildings.



\section{Objectives} 

The main objective of this thesis is the coupling of different scales in natural hazards. Usually, the action is originated in a far filed from where the the vulnerability is located. There are three scenarios with different scale. The action may be generated in a local or global scale, depending on the nature of the phenomena. The propagation is usually in the large scale and it introduces the need of using reduced models. Finally, the evaluation of the resilience of singular structures requires to descend to the local scale.

The objectives can be sumarized as
\begin{itemize}
    \item aaa
\end{itemize}

Gran escala: simplificaciones 2D a las ecuaciones de Navier-Stokes.

Escala local: fenómenos turbulentos que requieren la solución de las ecuaciones de Navier-Stokes en 3D con alta resolución.


\section{State of the art}


