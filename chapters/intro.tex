
\chapter{Introduction}
\label{chapter_introduction}



%%%%%%%%%%%%%%%%%%%%%%%%%%%%%%%%%%%%%%%%%%%%%%%%%%%%%%%%%%%%

\section{Prologue}


Floods are complex natural phenomena with a great assortment of causes and a huge casuistic
depending on the land characteristics or the climatic conditions. Preventing floods is still a great
challenge due to uncertainty of climatic conditions and complexity of modeling land, usually
defined by a great number of variables and parameters.
Some of the main flooding causes are torrential storms, hurricanes or monsoon, reservoir failure
or small floodplains. Near coastal zones floods are also caused by storm surges or even
tsunami waves. In the case of lakes and reservoirs, long waves may be generated by landslides.

Flooding is the most important natural hazard in terms causing of loss of life, displacement of
people and economic losses. But vulnerability can
increase when natural hazards make strategic structures fail, or when there aren't precautionary
measures or actuation plans.



Even though this study is about the numerical modelling of water related hazards and special attention has been paid to hydrological problems, the equations can be applied to a huge variety of problems. The configuration is modified just specifying the appropriate boundary conditions. Aside from the main scope of the thesis, some other examples are included. One example of application is related to pollutant transport in atmospheric environment. The other example is about tune liquid dampers for high rise buildings.



%%%%%%%%%%%%%%%%%%%%%%%%%%%%%%%%%%%%%%%%%%%%%%%%%%%%%%%%%%%%

\section{Objectives} 


The main objective of this thesis is the coupling of different scales in natural hazards. Usually, the action is originated in a far filed from where the the vulnerability is located. There are three scenarios with different scale. The action may be generated in a local or global scale, depending on the nature of the phenomena. The propagation is usually in the large scale and it introduces the need of using reduced models. Finally, the evaluation of the resilience of singular structures requires to descend to the local scale.

The objectives can be summarized as
\begin{itemize}
    \item aaa
\end{itemize}

Large scale: 2D simplifications for the Navier-Stokes equations.

Local scale: the turbulent flow require the solution of the Navier-Stokes equations with high fidelity resolution.



%%%%%%%%%%%%%%%%%%%%%%%%%%%%%%%%%%%%%%%%%%%%%%%%%%%%%%%%%%%%

\section{Organization of the thesis}


This thesis is structured in \textcolor{red}{\bfseries N} chapters, including the present introduction. Chapter \ref{equations} presents a review of the governing equations and the known methods used in the literature to obtain numerical approximations.
The bibliographic revision is specially extensive for the shallow water equations.

Chapter \ref{eulerian_sw} is dedicated to the Finite Element Method applied to the shallow water equations. Stabilized formulations and quasi-monotonic formulations are presented.

In Chapter \ref{lagrangian_sw} the Particle methods applied to the shallow water equations are presented. This family of methods present some interesting properties, such as the natural approximation of the shoreline. The inconvenient of the Particle methods is the lack of robustness for all the possible flow regimes.

\textcolor{red}{Chapter about mesh adaptivity?}

Chapter \ref{coupling} is devoted to the coupling strategies between the shallow water approximations and the fully resolved methods. An example of coupling for landslide long wave generation is provided.

Finally, two appendices are included to show another applications related to the shallow water equations, the particle methods and the large scale simulations.



%%%%%%%%%%%%%%%%%%%%%%%%%%%%%%%%%%%%%%%%%%%%%%%%%%%%%%%%%%%%

\section{Research dissemination}


Some of the developments if this thesis have been published in the format of articles in peer reviewed journals. Since the research has advanced gradually, the articles are related to a chapter, but there are some differences, which can be big.
The chapters are more extensive than the articles and some parts of the articles are omitted to avoid repetitions.
There is also not the same sequence between between the publication date and the chapter order.
On the other hand, since the notation is introduced gradually, it has been unified in the thesis and may be slightly different from the articles and this document.

\paragraph{Chapter \ref{eulerian_sw}} \fullcite{maso2022}
\paragraph{Chapter \ref{lagrangian_sw}} \fullcite{puigferrat2021}
\paragraph{Chapter \ref{coupling}} \fullcite{maso2022b}


